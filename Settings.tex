% Do not compile, settings document 

% ========== Classe du document ==========
\documentclass[12pt]{report}    % A4, police taille 11, report pour avoir les chapitres

% ========== Langue française pour le document ==========
\usepackage{latexsym}       % Police latex de base
\usepackage[french]{babel}  % Dictionnaire français (indentation, caractères spéciaux, tirets...)
\usepackage[utf8]{inputenc} % Encodage d'entrée pour les caractères accentués
\usepackage[T1]{fontenc}    % Affichage correct des caractères accentués notamment

% ========== Géométrie du document ==========
% Gestion des différentes marges du document pour coller avec le séparateur d'en-tête
\usepackage[left=2cm , right=2cm, bottom=2cm, top=2cm, headheight=2cm, headsep=0.5cm,heightrounded=true]{geometry}
% \raggedbottom % This makes the last line of each page be at exactly the same place on the sheet of paper
% \flushbottom  % All pages will not necessarily have exactly the same height, but ‘almost the same height’
\setlength{\parskip}{1em}   % Définition de l'espace entre les paragraphes
\setlength{\parindent}{4em} % Définition de la longueur du "tab" = indentation
\usepackage{fancyhdr}       % Permet en-tête et pied de page
\pagestyle{fancyplain}      % Pour avoir le même style sur les pages fancy et sur celles plains comme la toc
\renewcommand\plainheadrulewidth{.4pt}  % Le trait noir sous les logos sur les pages plain
\fancyhead[L]{\includegraphics[scale=0.1]{Images/Grenoble-Phelma-Logo.png}}    % Logo gauche
\fancyhead[R]{\includegraphics[scale=0.03]{Images/logo-soict-hust.png}}   % Logo droit
% Redéfinir le style "empty" utilisé par le documentclass "report" pour le titre, résumé et chapitres
\fancypagestyle{empty}
{
    \fancyhf{}
    \fancyhead[L]{\includegraphics[scale=0.1]{Images/Grenoble-Phelma-Logo.png}}    % Logo gauche
    \fancyhead[R]{\includegraphics[scale=0.03]{Images/logo-soict-hust.png}} % Logo droit
}


\usepackage{afterpage} % génère ds pages blanches
\newcommand\myemptypage{
    \null
    \thispagestyle{empty}
    \addtocounter{page}{1}
    \newpage
    }

% ========== Liens dans le document, méta-datas et références ==========
\usepackage{xpatch}     % Permet de patcher certaines fonctionnalités de base comme la toc
\usepackage{float}      % Placement des flottants
\usepackage{hyperref}   % Liens dans le documents
\hypersetup{pdfborder=0 0 0} % Pas d'encadré sur les liens
\usepackage{caption}    % Légendes dans les environnements "figure" et "float"
\usepackage[list=true]{subcaption} % Légendes pour les "sous-figures" et "sous-float"
                                   % et affichage des "sous-..." dans la liste des figures
% \def\thechapter{\Alph{chapter}}    % Définition des chapitres avec une lettre
\setcounter{tocdepth}{3}           % Profondeur de numérotation de la toc  (chap > sec > subsec > subsubsec)
\setcounter{secnumdepth}{3}        % Profondeur de numérotation des titres (chap > sec > subsec > subsubsec)
\usepackage{chngcntr}              % Permet de changer les numérotations d'objets
\usepackage[titles]{tocloft}       % Gestion très précise des différentes listes

\hypersetup % Attribution des méta-datas pdf pour reconnaissance automatique Zotero entre autres
{
pdftitle={Titre},
pdfsubject={Sujet},
pdfauthor={Auteur},
pdfkeywords={keyword1} {keyword2} {keyword3}
}

% ========== Graphique, police, maths ==========
\usepackage[table,xcdraw]{xcolor}       % Package permettant d'utiliser de la couleur
% \usepackage{color}                      % Rajouter de la couleur au texte
\usepackage{bm}                         % Mettre en gras des maths avec la commande \bm
\usepackage{ragged2e}                   % Meilleure gestion de l'alignement des textes entre autres
\usepackage{tcolorbox}                  % Boite colorées pour texte, images ou équations
\usepackage{textcomp}                   % Symboles et polices
\usepackage{gensymb}                    % Symbole pour le degré entre autre
\usepackage{amsmath,amsfonts,amssymb}   % Écrire des maths
\usepackage{cancel}                     % Barrer des maths
\usepackage{mathtools}                  % Gestion des matrices et de maths complexes
\usepackage{morewrites}                 % Résoud un problème entre les listes équation et codes


\usepackage{lmodern}
\usepackage[Lenny]{fncychap}



\ChNameUpperCase
\ChNumVar{\fontsize{40}{42}\usefont{OT1}{ptm}{m}{n}\selectfont}
\ChTitleVar{\Huge \bfseries}

\usepackage{pifont}
\usepackage{enumitem}       % Gestion des énumérations

\definecolor{bulletcolor}{RGB}{0,0,0}

% \setenumerate{label=\Alph*.}
\setenumerate{label=\arabic*.}
%\setenumerate{label=\roman*.}

\setitemize{label=\textbullet, font=\Large \color{bulletcolor}}


\usepackage{siunitx}                                    % Pour des unités bien écrites
\newcommand{\nomunit}[1]{%
\renewcommand{\nomentryend}{\hspace*{\fill}#1}}         % Commande pour la nomenclature (unités à droite)
\sisetup{inter-unit-product =\ensuremath{{}\cdot{}}}    % Séparation par un point des unités
\DeclareSIUnit\bar{bar}                                 % Besoin de déclarer les bar car pas pris en charge
\DeclareSIPower\quartic\tothefourth{4}

\usepackage{contour}
\usepackage[normalem]{ulem}

\renewcommand{\ULdepth}{1.8pt}
\contourlength{0.8pt}

\newcommand{\myuline}[1]{%
	\uline{\phantom{#1}}%
	\llap{\contour{white}{#1}}%
}

% myuline on each word to allow linebreaks
\RequirePackage{xparse}
\ExplSyntaxOn
\NewDocumentCommand{\myulineX}{m}
{
	\seq_set_split:Nnn \l_tmpa_seq { ~ } { #1 }
	\seq_map_inline:Nn \l_tmpa_seq { \myuline{##1} ~ } \unskip
}
\ExplSyntaxOff

% =========== Glossaire ============
\usepackage[acronym,toc,nonumberlist]{glossaries}% pas de numéro de page dans le glossaire


\newacronym{hust}{HUST}{Hanoi University of Science and Technology : Université d'accueil}
\newacronym{bkcs}{BKCS}{Bach Khoa Cyber Security Center : Laboratoire de cybersécurité de l'université que j'ai intégré}
\newacronym{bdd}{BDD}{Base de données}
\newacronym{onvif}{ONVIF}{Open Network Video Interface Forum : Standard de communication permettant l'interopérabilité entre appareils IP}
\newacronym{ia}{IA}{Intelligence Artificielle}                        % Récupère les informations du fichier glossary.tex
\makeglossaries                         % Génère le glossaire avec les informations récupérées

% ======= Diagramme de Gantt =======
\usepackage{pgfgantt}
\definecolor{color_Gantt}{RGB}{228, 5, 32}

% ===== Gestion des figures =========
\usepackage{graphicx}               % Plus d'arguments pour la fonction \includegraphics
\graphicspath{{Images/}}            % Path des images

\counterwithin{figure}{section}     % Numérotation des figures à partir des sections
\setcounter{lofdepth}{2}            % Afficher jusqu'aux sous-figures dans la liste des figures
\cftsetindents{figure}{0em}{3.5em}  % Réglage de l'espace entre le numéro et le nom de la figure dans la liste
\setlength\cftbeforefigskip{5pt}    % Réglage de l'espacement entre les figures dans la liste
\AtBeginDocument{\renewcommand{\listfigurename}{Liste des figures}} % Renommer la liste des figures
% Ajout de la position de la liste des figures dans la toc
\xpretocmd{\listoffigures}{\addcontentsline{toc}{chapter}{\listfigurename}}{}{}

% ===== Gestion des tableaux=========
\usepackage{array,multirow,makecell}                        % Packages utiles pour les tableaux
\setcellgapes{1pt}              \usepackage{booktabs}                         % Paramètres sympa d'après Xm1Math
\makegapedcells                                             % Paramètres sympa d'après Xm1Math
\newcolumntype{R}[1]{>{\raggedleft\arraybackslash }b{#1}}   % Paramètres sympa d'après Xm1Math
\newcolumntype{L}[1]{>{\raggedright\arraybackslash }b{#1}}  % Paramètres sympa d'après Xm1Math
\newcolumntype{C}[1]{>{\centering\arraybackslash }b{#1}}    % Paramètres sympa d'après Xm1Math

\counterwithin{table}{section}      % Numérotation des tableaux à partir des sections
\setcounter{lotdepth}{2}            % Afficher jusqu'aux sous-tableaux dans la liste des tableaux
\cftsetindents{table}{0em}{3.5em}   % Réglage de l'espace entre le numéro et le nom du tableau dans la liste
\setlength\cftbeforetabskip{5pt}    % Réglage de l'espacement entre les figures dans la liste
% Ajout de la position de la liste des tableaux dans la toc
\xpretocmd{\listoftables}{\addcontentsline{toc}{chapter}{\listtablename}}{}{}

% =============== Index =============
\usepackage{imakeidx}   % Package pour créer l'index
\makeindex              % Génération de l'index
% Ajout de la position de l'index dans la toc
\xpretocmd{\printindex}{\addcontentsline{toc}{chapter}{\indexname}}{}{}

% ========== Bibliographie ==========
% Importer un fichier biblatex, sans dépassement des marges, trié par ordre d'apparition
\usepackage[block=ragged,sorting=none]{biblatex}
\usepackage{csquotes}           % Gestion des caractères " " lors des citations
\addbibresource{biblio.bib}     % Importer le fichier de bibliographie
\nocite{*}                      % Importer les éléments non cités quand même dans la bibliographie

% ===== Gestion des annexes ==========
\usepackage[toc,page,title,titletoc,header]{appendix}   % Packages indexes importants
\usepackage{pdfpages}                                   % Intégration de pdf dans le document
\renewcommand{\appendixtocname}{Annexes}      % Nom de la table des annexes dans la toc
\renewcommand{\appendixpagename}{Annexes}               % Nom du titre de la page des annexes
\usepackage{titletoc}	% Permet de générer une petite table des annexes

% ========== Utilitaires =============
\usepackage[all,defaultlines=3]{nowidow}    % Macro pour la gestion des lignes seules en bout de page
\usepackage{blindtext}                      % Génération de texte aléatoire pour les exemples

% =================== Fin Préambule Latex ====================
